%!TEX root = ./main.tex

\begin{thebibliography}{99}

\bibitem{c1} S. C. Park, M. K. Park, and M. G. Kang, “Super-resolution image reconstruction:
A technical overview,” IEEE Signal Processing Magazine, vol. 20, no. 3, pp. 21–36,
May 2003.
\bibitem{keren} D. Keren, S. Peleg, and R. Brada, “Image sequence enhancement using subpixel
displacements,” in IEEE Computer Society Conference on Computer Vision and
Pattern Recognition, June 1988, pp. 742–746.
\bibitem{hardie} R. Hardie, K. Barnard, and E. Armstrong, “Joint MAP registration and high resolution image estimation using a sequence of undersampled images,” IEEE Transactions on Image Processing, vol. 6, no. 12, pp. 1621–1633, December 1997.
\bibitem{patti} A. Patti, M. Sezan, and A. Tekalp, “High-resolution image reconstruction from
a low-resolution image sequence in the presence of time-varying motion blur,” in
Proceedings of the IEEE International Conference on Image Processing, Austin,
TX, vol. 1, 1994, pp. 343–347.
\bibitem{hardie2} R. C. Hardie, K. J. Barnard, J. G. Bognar, E. E. Armstrong, and E. A. Watson, “High resolution image reconstruction from a sequence of rotated and translated
frames and its application to an infrared imaging system,” Optical Engineering,
vol. 37, no. 1, pp. 247–260, January 1998.
\bibitem{alam} M. S. Alam, J. G. Bognar, R. C. Hardie, and B. J. Yasuda, “Infrared image registration and high-resolution reconstruction using multiple translationally shifted aliased video frames,” IEEE Transactions on Instrumentation and Measurement, vol. 49,
no. 5, pp. 923–915, October 2000.
\bibitem{tsai} R. Y. Tsai and T. S. Huang, “Multiframe image restoration and registration,” in Advances in Computer Vision and Image Processing: Image Reconstruction from Incomplete Observations, T. S. Huang, Ed., vol. 1. London: JAI Press, 1984, pp. 317–339.
\bibitem{kim} N. K. Bose, H. C. Kim, and H. M. Valenzuela, “Recursive total least squares algorithm for image reconstruction from noisy undersampled frames,” Multidimensional Systems and Signal Processing, vol. 4, no. 3, pp. 253–268, July 1993.
\bibitem{yang} J. Yang, J. Wright, T. Huang, and Yi Ma. Image super-resolution via sparse representation. IEEE Transactions on Image Processing (TIP), vol. 19, issue 11, 2010.
\bibitem{ransac} Martin A. Fischler and Robert C. Bolles (June 1981). “Random Sample Consensus: A Paradigm for Model Fitting with Applications to Image Analysis and Automated Cartography”. Comm. of the ACM 24 (6): 381–395
\bibitem{wired_ff_algo} Jordan Ellenberg, ”Fill in the Blanks: Using Math to Turn Lo-Res Datasets Into Hi-Res Samples”, 22 February 2010, Wired Magazine \url{http://www.wired.com/2010/02/ff_algorithm/}
\bibitem{incoherence} D.L. Donoho and X. Huo, “Uncertainty principles and ideal atomic decomposition,” IEEE Trans. Inform. Theory, vol. 47, no. 7, pp. 2845–2862, Nov. 2001.
\bibitem{candes} E. Candes, “Compressive sensing,” in Proceedings of the International Congress of Mathematicians, vol. 3, pp. 1433–1452, 2006.
\bibitem{donoho} D. L. Donoho, “Compressed sensing,” IEEE Transactions on Information Theory, vol. 52, no. 4, pp. 1289–1306, 2006.
\bibitem{proakis_sampling} John G. Proakis, Dimitris G. Manolakis , “Digital Signal Processing - Principles, Algorithms, Implementations” 4th edition (Pearson International Edition), Pearson Education, Chapter 1.4.2: The Sampling Theorem 
\bibitem{gonzalez_2d_fft} Rafael C. Gonzalez, Richard E. Woods , “Digital Image Processing” 3rd edition (Pearson International Edition), Pearson Education, Chapter 1.4.2: The 2D Discrete Fourier Transform and its Inverse
\bibitem{shift_add_fusion} S. Farsiu, M. Elad, P. Milanfar, “Fast and Robust Multiframe Super Resolution” IEEE Transactions on Image Processing, vol. 13, no. 10, October 2004
\bibitem{imregintro} ”Image Registration", in Wikipedia: The Free Encyclopedia; (Wikimedia Foundation Inc., updated 2 December 2014, 15:50 UTC) \url{http://en.wikipedia.org/wiki/Image_registration}
\bibitem{cs_intro} M. Davenport, M. Duarte, Y. Eldar, G. Kutyniok, ”Introduction to Compressed Sensing”, Stanford University - Department of Statistics
\bibitem{convexmin} F. Bach, R. Jenatton, J. Mairal, G. Obozinski, ”Convex Optimization with Sparsity-Inducing Norms”, Institut National de Recherche en Informatique et en Automatique (INRIA)
\bibitem{red_dict} H. Rauhut, K. Schnass, and P. Vandergheynst, “Compressed sensing and redundant dictionaries,” IEEE Transactions on Information Theory, vol. 54, no. 5, May 2008.
\bibitem{denoising_dict} M. Elad and M. Aharon, “Image denoising via sparse and redundant representations over learned dictionaries,” IEEE Transactions on Image Processing, vol. 15, pp. 3736–3745, 2006.
\bibitem{im_vid_rest} J. Mairal, G. Sapiro, and M. Elad, “Learning multiscale sparse representations for image and video restoration,” Multiscale Modeling and Simulation, vol. 7, pp. 214–241, 2008.
\bibitem{ksvd_dict} M. Aharon, M. Elad, and A. Bruckstein, “K-SVD: An algorithm for designing overcomplete dictionaries for sparse representation,” IEEE Transactions on Signal Processing, vol. 54, no. 11, pp. 4311–4322, Nov. 2006.
\bibitem{sparse_coding_dict} H. Lee, A. Battle, R. Raina, and A. Y. Ng, “Efficient sparse coding algorithms,” in Advances in Neural Information Processing Systems (NIPS), pp. 801–808, 2007.
\bibitem{human_sparse} B. Olshausen and D. Field, “Sparse coding with an overcomplete basis set: A strategy employed by V1?,” Vision Research, vol. 37, no. 23, pp. 3311–3325, 1997.
\bibitem{nphard} ”NP-hard", in Wikipedia: The Free Encyclopedia; (Wikimedia Foundation Inc., updated 9 April 2015, 15:30 UTC) \url{http://en.wikipedia.org/wiki/NP-hard}
\bibitem{nphard1} D. L. Donoho, “For most large underdetermined systems of linear equations, the minimal $l_1$-norm solution is also the sparsest solution,” Communications on Pure and Applied Mathematics, vol. 59, no. 6, pp. 797–829, 2006.
\bibitem{nphard2} “For most large underdetermined systems of linear equations, the minimal $l_1$-norm near-solution approximates the sparsest near-solution,” Communications on Pure and Applied Mathematics, vol. 59, no. 7, pp. 907–934, 2006.
\bibitem{lasso} R. Tibshirani, “Regression shrinkage and selection via the lasso,” Jour- nal of Royal Statistical Society, Series B, vol. 58, no. 1, 1996.
\bibitem{dict_training} Jianchao Yang, Zhaowen Wang, Zhe Lin, and Thomas Huang. Coupled dictionary training for image super-resolution. IEEE Transactions on Image Processing (TIP), vol. 21, issue 8, pages 3467-3478, 2012.
\bibitem{srexample} W. T. Freeman, T. R. Jones, and E. C. Pasztor, “Example-based superresolution,” IEEE Computer Graphics and Applications, vol. 22, pp. 56–65, 2002.
\bibitem{lowlevel} W. T. Freeman, E. C. Pasztor, and O. T. Carmichael, “Learning low-level vision,” International Journal of Computer Vision, vol. 40, no. 1, pp. 25–47, 2000.
\bibitem{sr_neigh} H. Chang, D.-Y. Yeung, and Y. Xiong, “Super-resolution through neighbor embedding,” in IEEE Conference on Computer Vision and Pattern Classifi- cation (CVPR), vol. 1, pp. 275–282, 2004.
\bibitem{imhal} J. Sun, N. N. Zheng, H. Tao, and H. Shum, “Image hallucination with primal sketch priors,” in IEEE Conference on Computer Vision and Pattern Recognition (CVPR), vol. 2, 2003, pp. 729–736.
\bibitem{imfill} imfill, Image Processing Toolbox Documentation, MathWorks \\ \url{http://www.mathworks.com/help/images/ref/imfill.html}
\bibitem{imresize} imresize, Image Processing Toolbox Documentation, MathWorks \\ \url{http://www.mathworks.com/help/images/ref/imresize.html}
\bibitem{k-term} A. Cohen, W. Dahmen, R. DeVorce, "Compressed sensing and best k-term approximation", in the Journal of the American Mathematical Society, vol. 22, pp. 211-231 , 2009.




\end{thebibliography}

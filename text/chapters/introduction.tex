%!TEX root = ../main.tex

\chapter*{Εισαγωγή}
\markboth{Εισαγωγη}{}
%\vspace{-1.3in}
\lettrine[findent=2pt]{\fbox{\textbf{Η}}}{} δημιουργία ενός Συστήματος Συστάσεων (Recommender System – RS) απαιτεί καλή γνώση του αντικειμένου το οποίο αφορά ο recommender. Γι’ αυτό είναι σημαντικό ένας μηχανικός να μπορεί να δημιουργήσει γρήγορα «διαισθήσεις» πάνω στα δεδομένα στα οποία δουλεύει. Η χρήση απεικονίσεων κατά τη διάρκεια του σχεδιασμού και υλοποίησης του RS μπορεί να βοηθήσει σε μεγάλο βαθμό την εξοικείωση του μηχανικού με τα δεδομένα και το πεδίο στο οποίο δουλεύει. Συγκεκριμένα, σ’ αυτή τη διπλωματική ασχοληθήκαμε με απεικονίσεις που σκοπεύουν να βοηθήσουν τον μηχανικό στη διαδικασία επιλογής χαρακτηριστικών για τον υπολογισμό ψευδο-αξιολογήσεων (pseudo-ratings) για ένα Σύστημα Συστάσεων Συνεργατικού Φιλτραρίσματος με Έμμεσες Αξιολογήσεις (Collaborative Filtering with Implicit Feedback). 
